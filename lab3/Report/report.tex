\documentclass[12pt, a4paper]{article}
\usepackage{polski}
\usepackage{algorithmic}
\usepackage{mathtools}
\usepackage{multicol}
\usepackage[utf8]{inputenc}
\begin{document}
\linespread{1.3}
\title{Bezpieczeństwo komputerowe - Lista 2}
\author{Kacper Zieliński 236698}
\date{}

\maketitle
\section*{2. Podłączone urządzenia}
\begin{table}[h]
    \centering
    \footnotesize
    \begin{tabular}{l | l}
        Nazwa sieci & Liczba urządzeń \\ \hline
        Pasaż Grunwaldzki Free Wifi & 6 \\
        \end{tabular}
    \caption{Ilośc podłączonych urządzeń do publicznej sieci na podstawie wire001.pcapng}
\end{table}

\section*{3. Lista odwiedzonych stron www}
\begin{multicols}{2}
    \begin{itemize}
        \item google.pl 
        \item google.com
        \item doubleclick.net 
        \item creativecdn.com 
        \item facebook.net 
        \item gemius.pl 
        \item facebook.com
        \item gry.pl 
        \item smarturl.it 
        \item ourgames.ru 
        \item youtube.com 
        \item live.com 
        \item gstatic.com 
        \item microsoft.com
        \item apple.com 
        \item icloud.com
        \item moovitapp.com 
        \item digicert.com
        \item viber.com 
        \item appboy.com 
        \item cloudfront.net 
        \item chem.latech.edu
        \item thoughtco.com 
        \item casalmedia.com 
        \item gamesgames.com 
        \item jeux.fr 
    \end{itemize}
\end{multicols}
\vspace{5mm}
Lista odwiedzonych stron WWW przez użytkowników.

\section*{4. Lista protokołów}
\begin{multicols}{2}
    \begin{itemize}
        \item XID
        \item UDP
        \item TCP
        \item DHCP
        \item TLSv1.2
        \item HTTP
        \item GQUIC
        \item DNS
        \item ARP
        \item SSDP
        \item OCSP
        \item NTP
        \item NBNS
        \item MPTCP
        \item MDNS
        \item LLMNR
        \item IPX
        \item IGMPv3
        \item ICMPv6
        \item ICMP
    \end{itemize}
\end{multicols}
\vspace{5mm}
Poprzez nieszyfrowane połączenie możemy podejrzec stronę www.gry.pl
\clearpage
\section*{5. Lokalizacja adresów}
\begin{table}[h]
    \centering
    \footnotesize
    \begin{tabular}{l | l}
        Adres IP & Lokalizacja \\ \hline
        2.16.127.9 & United Kingdom, England, London \\
        104.199.64.120 & United States,	California,	Mountain View \\
        78.11.56.202 & Poland, Lubelskie, Leczna \\
        52.76.148.145 & Singapore, Singapore, Singapore \\
        172.217.20.195 & Netherlands, 	Noord-Holland,	Amsterdam  \\
        54.230.228.77 & United States, 	Washington,	Seattle \\
        34.250.22.111 &  Ireland, 	Leinster,	Dublin\\
        146.185.181.89 & Netherlands, 	Noord-Holland,	Amsterdam \\
        212.2.121.155 & Poland, 	Mazowieckie,	Warsaw (Mokotów)  \\
        93.184.221.240 & United Kingdom, England,	London \\
        93.184.220.29 & United Kingdom, 	England,	London \\
        62.129.196.99 & Poland, 	Zachodniopomorskie,	Szczecin \\
        94.23.89.14 & France, 	Île-de-France,	Paris\\
        205.185.216.42 & United States, 	Texas,	Dallas  \\
        152.199.20.178 & United States, 	Virginia,	Ashburn \\
        172.217.16.33 & United States, 	California,	Mountain View \\
        \end{tabular}
    \caption{Adresy IP i ich lokalizacje}
\end{table}

\end{document}